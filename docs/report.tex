%%% Для сборки выполнить 2 раза команду: pdflatex <имя файла>

\documentclass[a4paper,12pt]{article}

\usepackage{ucs}
\usepackage[utf8x]{inputenc}
\usepackage[russian]{babel}
%\usepackage{cmlgc}
\usepackage{graphicx}
\usepackage{listings}
\usepackage{xcolor}
\usepackage{titlesec}
%\usepackage{courier}


\makeatletter
\renewcommand\@biblabel[1]{#1.}
\makeatother

\newcommand{\myrule}[1]{\rule{#1}{0.4pt}}
\newcommand{\sign}[2][~]{{\small\myrule{#2}\\[-0.7em]\makebox[#2]{\it #1}}}

% Поля
\usepackage[top=20mm, left=30mm, right=10mm, bottom=20mm, nohead]{geometry}
\usepackage{indentfirst}

% Межстрочный интервал
\renewcommand{\baselinestretch}{1.50}

% ------------------------------------------------------------------------------
% minted
% ------------------------------------------------------------------------------
% \usepackage{minted}


% ------------------------------------------------------------------------------
% tcolorbox / tcblisting
% ------------------------------------------------------------------------------
\usepackage{xcolor}
\definecolor{codecolor}{HTML}{FFC300}

\usepackage{tcolorbox}
% \tcbuselibrary{most,listingsutf8,minted}

\tcbset{tcbox width=auto,left=1mm,top=1mm,bottom=1mm,
right=1mm,boxsep=1mm,middle=1pt}

% \newtcblisting{myr}[1]{colback=codecolor!5,colframe=codecolor!80!black,listing only, 
% minted options={numbers=left, style=tcblatex,fontsize=\tiny,breaklines,autogobble,linenos,numbersep=3mm},
% left=5mm,enhanced,
% title=#1, fonttitle=\bfseries,
% listing engine=minted,minted language=r}

%%%%%%%%%%%%%%%%%%%%%%%%%%%%%%%%%%%%%%%

\begin{document}

%%%%%%%%%%%%%%%%%%%%%%%%%%%%%%%
%%%                         %%%
%%% Начало титульного листа %%%

\thispagestyle{empty}
\begin{center}


    \renewcommand{\baselinestretch}{1}
    {\large
        {\sc Петрозаводский государственный университет\\
            Институт математики и информационных технологий\\
            Кафедра информатики и математического обеспечения
        }
    }

\end{center}


\begin{center}
    %%%%%%%%%%%%%%%%%%%%%%%%%
    %
    % Раскомментируйте (уберите знак процента в начале строки)
    % для одной из строк типа направления  - бакалавриат/
    % магистратура и для одной из
    % строк Вашего направление подготовки
    %
    Направление подготовки бакалавриата \\
    % 01.03.02 --- Прикладная математика и информатика \\
    % 09.03.02 --- Информационные системы и технологии \\
    09.03.04 --- Прикладная математика и информатика \\
    %%%%%%%%%%%%%%%%%%%%%%%%%
\end{center}

\vfill

\begin{center}

    {\normalsize
        Отчет о проектной работе по курсу <<Основы информатики и программирования>>}
    \medskip


    {\Large \sc {Разработка приложения \\ <<Тетрис>> }} \\
\end{center}

\medskip

\begin{flushright}
    \parbox{11cm}{%
        \renewcommand{\baselinestretch}{1.2}
        \normalsize
        Выполнил:\\

        студент 1 курса группы 22104
        \begin{flushright}
            Красников Евгений Александрович \sign[подпись]{4cm}
        \end{flushright}


    }
\end{flushright}

\vfill

\begin{center}
    \large
    Петрозаводск --- 2021
\end{center}

\newpage

\tableofcontents

\newpage
\section*{Введение}
\addcontentsline{toc}{section}{Введение}

Цель проекта: разработать игру <<Тетрис>> на С++ 

Задачи проекта:
%%% Пример создания списков %%%
\begin{enumerate}
    \item Создать интерфейс
    \item Реализовать классы C++ для описания объектов игры
    \item Реализовать управление в игре (горизонтальное и вертикальное перемещение фигур, поворот и ускорение их падения)
    \item Реализовать механизм уничтожения при получении ровных линий из фигур
    \item Реализовать логику игры
\end{enumerate}

%%%                          %%%
%%%%%%%%%%%%%%%%%%%%%%%%%%%%%%%%

%%%%%%%%%%%%%%%%%%%%%%%%%%%%%%%
%%% Требования к приложению %%%
% \newpage
\section{Требования к приложению}
\begin{itemize}
    \item Приятный графический интерфейс
    \item Функционал игры <<Тетрис>> 
    \item Возможность управления игрой с клавиатуры
    \item Возможность перезапуска игры
\end{itemize}
% \subsection{Подраздел}

%%%                                     %%%
%%%%%%%%%%%%%%%%%%%%%%%%%%%%%%%%%%%%%%%%%%%

%%%%%%%%%%%%%%%%%%%%%%%%%%%%%%%%%%%%%%%%%%%
%%%                                     %%%
%%% Проектирование приложения           %%%
% \newpage
\section{Проектирование приложения}

Модули приложения:

\begin{enumerate}
    \item Tetris.pro - файл описания проекта
    \item main.cpp - точка запуска, в которой создается и выводится на экран виджет;
    \item tetrismodel.(h|cpp) - реализация Модели с логикой игры.
    \item tetriscontroller.(h|cpp) - реализация Контроллера. Здесь находится логика управления
    \item tetrisview.(h|cpp) - реализация Представления. Вывод графики реализован стандартными средствами Qt на основе QPainter.
\end{enumerate}

%%%                          %%%
%%%%%%%%%%%%%%%%%%%%%%%%%%%%%%%%

%%%%%%%%%%%%%%%%%%%%%%%%%%%%%%%%
%%%                          %%%
%%% Реализация приложения    %%%
% \newpage
\section{Реализация приложения}

Для реализации игры был использован язык <<C++>> . В приложении активно используются стандартные билиотеки Qt -- QRect, QPoint, QSize, QTimer и др..
\begin{itemize}
    \item Количество модулей: 8
    \item Количество классов: 6
    \item Количество C++ функций: 46
    \item Количество строк C++ кода: 710
\end{itemize}

%%%                          %%%
%%%%%%%%%%%%%%%%%%%%%%%%%%%%%%%%

%%%%%%%%%%%%%%%%%%%%%%%%%%%%%%%%
%%%                          %%%
%%% Заключение               %%%

\newpage
\section*{Заключение}
\addcontentsline{toc}{section}{Заключение}
В итоге получено приложение-игра <<Тетрис>>, удалось реализовать все запланированные функции. Получен новый опыт создания компьютерных игр и работы с QtCreator.

Финальный вид интерфейса:
\begin{center}
    \includegraphics[scale=0.8]{e0RVO4Xqzzs.jpg}
\end{center}

%%%                          %%%
%%%%%%%%%%%%%%%%%%%%%%%%%%%%%%%%

%%%%%%%%%%%%%%%%%%%%%%%%%%%%%%%%
%%%                          %%%
%%% Приложение               %%%

\newpage
\appendix
%\section*{Приложение}
%\addcontentsline{toc}{section}{Приложение}
%\titleformat{\section}[display]
%  {\normalfont\Large\bfseries}
%  {Приложение\ \thesection}
%  {0pt}{\Large\centering}
%\renewcommand{\thesection}{\Asbuk{section}}

%%%                          %%%
%%%%%%%%%%%%%%%%%%%%%%%%%%%%%%%%
\end{document}
