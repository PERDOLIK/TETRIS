\documentclass[9pt]{beamer}
\usepackage[T1,T2A]{fontenc}
\usepackage[utf8]{inputenc}
\usepackage{hyperref}
\hypersetup{unicode=true}
\usepackage{fontawesome}
\usepackage{graphicx}
\usepackage{bookmark}
\usepackage[english,russian]{babel}

\usetheme{CambridgeUS}
\usecolortheme{beaver}

\title{Разработка приложения <<Тетрис>>}
\subtitle{Отчет о проектной работе по курсу <<Основы информатики и программирования>>}
\author{Красников Евгений Александрович}
\date{28 июня 2021}

\begin{document}

\maketitle

\begin{frame}[fragile]{Цель проекта}
    Цель работы -- разработать игру <<Тетрис>> на C++. Получить навыки создания собственных проектов с графическим интерфейсом.
\end{frame}

\begin{frame}
    \frametitle{Этапы разработки приложения}
    \begin{itemize}
        \item Создать интерфейс
        \item Реализовать классы C++ для описания объектов игры
        \item Реализовать управление в игре (горизонтальное и вертикальное перемещение фигур, поворот и ускорение их падения)
        \item Реализовать механизм уничтожения при получении ровных линий из фигур
        \item Реализовать логику игры
    \end{itemize}
\end{frame}

\begin{frame}[fragile]{Результат}
    Реализовано:
    \begin{itemize}
        \item Рамки игрового поля
        \item Управление игрой с клавиатуры
        \item Механизм уничтожения при получении ровных линий из фигур
        \item Обработка завершения игры (если игровое поле переполняется, игра начинается заново)
    \end{itemize}

    \begin{center}
        \includegraphics[scale=0.25]{e0RVO4Xqzzs.jpg}
    \end{center}

\end{frame}

\begin{frame}[fragile]{Результат}
    Разработана игра Тетрис. Получилось реализовать всё, что было запланировано.\\
    Что можно добавить и улучшить:
    \begin{itemize}
        \item Отображение игровых очков
        \item Функцию подсчета очков
        \item Разные уровни сложности
    \end{itemize}
\end{frame}

\begin{frame}
    \begin{center}
        \Large
        \underline{Спасибо за внимание!}
    \end{center}
\end{frame}

\end{document}